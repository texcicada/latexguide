\documentclass{article}
\usepackage{fontspec}
\setmainfont{Noto Serif}
\setsansfont{Noto Sans}
\setfontface\fcommand{Hoefler Txt}
\usepackage{tcolorbox}
\tcbuselibrary{skins,listings,breakable}
\usetikzlibrary{shadows,shapes.symbols}
%\usepackage{hyperref}
	

\newcommand\fred[1]{\textrm{\textcolor{red}{{\sffamily\large#1}}}}

\newcommand\fredd[1]{\fred{\{}#1\fred{\}}}

\newcommand\frede[1]{\textbf{\textsf{\textcolor{blue}{\textbackslash#1}}}}

\newcommand\seename{see}
\newcommand\labelseeref[1]{\label{\seename:#1}}
\newcommand\seeref[1]{\textit{#1}\textsuperscript{on p \pageref{\seename:#1}}}
\newcommand\see[1]{see \seeref{#1}}
\newcommand\See[1]{\ \hfill\small See \seeref{#1}}

%===== democode: counter and label
\newcounter{democodecounter}
\newcommand\dccset{%
\refstepcounter{democodecounter}%
\thedemocodecounter%
\label{dcc:\thedemocodecounter}%
}

%===== democode: title and number/textlabel
\newcommand\dccodetitle[2]{%
#1\hfill c.\dccset\label{dcc:#2}%
}

%===== democode: ref
\newcommand\getdccref[1]{\textbf{c.\ref{dcc:#1}}}
\newcommand\getdccrefp[1]{code example \textbf{c.\ref{dcc:#1}}\textsuperscript{on p \pageref{dcc:#1}}}



%===== keywords
\lstdefinelanguage[LaTeX]{TeX}
{morekeywords={colorbox,
textcolor,
fbox,
mbox,
raggedright,			
sffamily,
ttfamily,
rmfamily,
textit,
textbf,
texttt,
textsf,
textsc,
bfseries,
mdseries,
itshape,
upshape,
fcolorbox,
setlength,
begin,
item,
end,
section,
},
%sensitive=false,
%morecomment=[l]{//},
%morecomment=[s]{/*}{*/},
%morestring=[b]",
}



%===== codeline label
\newcounter{clcounter}
\newcommand\cllabel{%
\refstepcounter{clcounter}%
\marginpar{\ \\(\theclcounter)}%
\label{cl:\theclcounter}%
}


%===== codeline environment
\newenvironment{codeline}[1]{%
	\begin{center}
		\ttfamily\cllabel\label{cl:#1}
}%		\lstinline!\textit{italic text}! 
{	\end{center}\ignorespaces
}



%===== codeline ref
\newcommand\getclref[1]{(\ref{cl:#1})}
\newcommand\getclpref[1]{\pageref{cl:#1}}
\newcommand\getclGen[1]{%
Codeline \getclref{#1}\textsuperscript{on p \getclpref{#1}}%
}
\newcommand\getclgen[1]{%
codeline \getclref{#1}\textsuperscript{on p \getclpref{#1}}%
}




%========================================
\begin{document}\begin{tcolorbox}[enhanced,width=2.5cm,
%square,circular arc,
square,octogon arc,
arc is angular,%cloud,
halign=center,valign=center,
colback=red!5!white,colframe=red!75!black,
%frame code={\node[cloud, draw, fill=blue!20, aspect=2] {ABC};},
frame style={left color=red!75!black,
right color=blue!75!black}
 ]
tcolorbox
\end{tcolorbox}

\begin{tcolorbox}[enhanced,
size=minimal,auto outer arc,
width=2.1cm,octogon arc,
colback=red,colframe=white,colupper=white,
fontupper=\fontsize{7mm}{7mm}\selectfont\bfseries\sffamily,
halign=center,valign=center,
square,arc is angular,
borderline={0.2mm}{-1mm}{red} ]
STOP
\end{tcolorbox}
\begin{tcolorbox}[enhanced,
size=minimal,auto outer arc,
width=2.1cm,octogon arc,
colback=red,colframe=white,colupper=white,
fontupper=\fontsize{7mm}{7mm}\selectfont\bfseries\sffamily,
halign=center,valign=center,
square,arc is angular,
borderline={0.2mm}{-1mm}{red} ]
СТОП
\end{tcolorbox}


\begin{tcolorbox}[tikz upper,title=TikZ drawing,colframe=blue]
\node[cloud, draw, fill=blue!20, aspect=2] {Cloud};
\end{tcolorbox}

%\begin{tcolorbox}[title=fcolorbox,colframe=blue,fontlower=\fcommand,]
%Draw a coloured frame around text on a coloured background.
%\tcblower
%\frede{fcolorbox}\fredd{frame-colour}\fredd{text-background-colour}\fredd{text}
%\end{tcolorbox}


%\begin{tcolorbox}[title=textcolor,colframe=blue,fontlower=\fcommand,]
%Typeset coloured text.
%\tcblower
%\frede{textcolor}\fredd{text-colour}\fredd{text}
%\end{tcolorbox}



\begin{tcblisting}{tikz lower,listing side text,fonttitle=\bfseries,
bicolor,colback=blue!50!white,colbacklower=white,colframe=black,
righthand width=3cm,title=TikZ\ drawing}
\path[fill=yellow,draw=yellow!75!red]
(0,0) circle (1cm);
\fill[red] (45:5mm) circle (1mm);
\fill[red] (135:5mm) circle (1mm);
\draw[line width=1mm,red]
(215:5mm) arc (215:325:5mm);
\end{tcblisting}




%===============================
%pre-sets
\tcbset{codedemo/.style={listing side text,fonttitle=\bfseries\sffamily,
bicolor,colback=blue!12,colbacklower=yellow!72!green!32,colframe=black,
righthand width=3cm}}

\tcbset{bluesyntax/.style={%
 colframe=blue,fontlower=\fcommand,
}}

\lstset{%
language={[LaTeX]TeX},
keywordstyle=\color{blue}\bfseries,
}



%----------------------------------------------
%=== Text formatting
\begin{tcolorbox}[bluesyntax,title=textformat,]
Typeset text as bold, italic, or both.\labelseeref{textformat}

\See{textcolor}.
\tcblower
\frede{textbf}\fredd{text}\\
\frede{textit}\fredd{text}\\
\frede{bfseries}\\
\frede{mdseries}\\
\frede{itshape}\\
\frede{upshape}\\
\par\medskip
\begin{tcblisting}{codedemo,title={\dccodetitle{Text formatting -- commands}{textcommands}}}
\raggedright Normal text, \textbf{bold text}, \textit{italic text}, \textbf{\textit{bold italic text}}. Back to normal text. %\textsc{Small Caps}
\end{tcblisting}
\begin{tcblisting}{codedemo,title={\dccodetitle{Text formatting -- switches}{textswitches}}}
\raggedright Normal text, \bfseries bold text, \itshape bold italic text, \upshape back to upright bold, \mdseries \itshape now some italics, \upshape and back to normal text.
\end{tcblisting}
\end{tcolorbox}


%\begin{tcblisting}{codedemo,title={\dccodetitle{Text formatting -- commands}{textcommands}}}
%\raggedright Normal text, \textbf{bold text}, \textit{italic text}, \textbf{\textit{bold italic text}}. Back to normal text.
%\end{tcblisting}

The command
%	\fbox{\parbox{0.8\linewidth}{%
	\begin{codeline}{textit}
%		\ttfamily
		\lstinline!\textit{italic text}! 
	\end{codeline}
%	}}
switches to italics locally: 
	\begin{codeline}{local}
		\lstinline!\textit[1]{{\itshape#1}}!
	\end{codeline}


%\begin{tcblisting}{codedemo,title={\dccodetitle{Text formatting -- switches}{textswitches}}}
%\raggedright Normal text, \bfseries bold text, \itshape bold italic text, \upshape back to upright bold, \mdseries \itshape now some italics, \upshape and back to normal text.
%\end{tcblisting}




%----------------------------------------------
%=== Font families
\begin{tcolorbox}[bluesyntax,title=fontfamily,]
Set text in a paricular font family: serif (roman), sans-serif, mono-spaced.\labelseeref{fontfamily}

%\See{xxx}.
\tcblower
%\frede{textrm}\fredd{text}\\
\frede{textsf}\fredd{text}\\
\frede{texttt}\fredd{text}\\
\frede{rmfamily}\\
\frede{sffamily}\\
\frede{ttfamily}\\
\par\medskip
\begin{tcblisting}{codedemo,title={\dccodetitle{Font families -- commands}{fontfamilycommands}}}
\raggedright Serif text (roman) abfgy 123, \textsf{sans text abfgy 123}, \texttt{mono-spaced text abfgy 123}, back to normal text.
\end{tcblisting}
\begin{tcblisting}{codedemo,title={\dccodetitle{Font families -- switches}{fontfamilyswitches}}}
\raggedright Serif text (roman) abfgy 123, \sffamily sans text abfgy 123, \ttfamily mono-spaced text abfgy 123, \rmfamily back to normal text.
\end{tcblisting}
\end{tcolorbox}





\begin{tcblisting}{codedemo,title={\dccodetitle{Boxes -- mbox}{mbox}}}
Some text \mbox{a a a a a a a} b b b b b b b b b b b b b b 
\end{tcblisting}

\begin{tcblisting}{codedemo,title={\dccodetitle{Boxes -- fbox}{fbox}}}
\raggedright Some \fbox{text} in a frame.
\end{tcblisting}


\begin{tcolorbox}[bluesyntax,title=colorbox,]
Put text on a coloured background.\labelseeref{colorbox}

\See{fcolorbox}.
\tcblower
\frede{colorbox}\fredd{text-background-colour}\fredd{text}\par\medskip
\begin{tcblisting}{codedemo,title={\dccodetitle{Boxes -- colorbox}{colorbox}}}
\raggedright Some coloured \colorbox{yellow!50}{ \textcolor{red}{text} }.
\end{tcblisting}
\end{tcolorbox}
See \getdccrefp{fcolorbox} for the framed version.

%\begin{tcblisting}{codedemo,title={\dccodetitle{Boxes -- fcolorbox}{fcolorbox}}}
%\raggedright Some framed \& coloured \fcolorbox{blue}{yellow!50}{text}.
%\end{tcblisting}





\begin{tcolorbox}[bluesyntax,title=fcolorbox,]
Draw a coloured frame around text on a coloured background.\labelseeref{fcolorbox}

\See{colorbox}.
\tcblower
\frede{fcolorbox}\fredd{frame-colour}\fredd{text-background-colour}\fredd{text}\par\medskip
\begin{tcblisting}{codedemo,title={\dccodetitle{Boxes -- fcolorbox}{fcolorbox}}}
\raggedright Some framed \& coloured \fcolorbox{blue}{yellow!50}{text}.
\end{tcblisting}
\end{tcolorbox}



%\begin{tcblisting}{codedemo,title={\dccodetitle{Boxes -- fboxrule}{fboxrule}}}
%\setlength{\fboxrule}{3pt}
%\raggedright Some framed \& coloured \fcolorbox{blue}{yellow!50}{text}.
%\end{tcblisting}

%\begin{tcblisting}{codedemo,title={\dccodetitle{Boxes -- fboxsep}{fboxsep}}}
%\setlength{\fboxrule}{3pt}
%\setlength{\fboxsep}{0pt}
%\raggedright Some framed \& coloured \fcolorbox{blue}{yellow!50}{text}.
%\end{tcblisting}




\begin{tcolorbox}[title=textcolor,colframe=blue,fontlower=\fcommand,]
Typeset coloured text.\labelseeref{textcolor}

\See{textformat}.
\tcblower
\frede{textcolor}\fredd{text-colour}\fredd{text}\par\medskip
\begin{tcblisting}{codedemo,title={\dccodetitle{Text -- textcolor}{textcolor}}}
\raggedright Some coloured \textcolor{red}{red text} and some \textcolor{blue}{blue text}.
\end{tcblisting}
\end{tcolorbox}



\begin{tcolorbox}[enhanced,breakable,title=adjusting boxes,colframe=blue,fontlower=\fcommand,]
Change the rule width, and the separator width between rule and text.
\tcblower
\frede{setlength}\fredd{\frede{fboxrule}}\fredd{length}\\\frede{setlength}\fredd{\frede{fboxsep}}\fredd{length}\par\medskip
\begin{tcblisting}{codedemo,title={\dccodetitle{Boxes -- fboxrule}{fboxrule}}}
\setlength{\fboxrule}{3pt}
\raggedright Some framed \& coloured \fcolorbox{blue}{yellow!50}{%
\textcolor{red}{text}}.
\end{tcblisting}
\begin{tcblisting}{codedemo,title={\dccodetitle{Boxes -- fboxsep}{fboxsep}}}
\setlength{\fboxrule}{3pt}
\setlength{\fboxsep}{0pt}
\raggedright Some framed \& coloured \fcolorbox{blue}{yellow!50}{%
\textcolor{red}{text}}.
\end{tcblisting}
\end{tcolorbox}







%\begin{tcblisting}{codedemo,title={\dccodetitle{Boxes -- fboxsep}{fboxsep}}}
%\setlength{\fboxrule}{-10pt}
%%\setlength{\fboxsep}{-10pt}
%\raggedright Some framed \& coloured \fcolorbox{blue}{yellow!50}{text}.
%\end{tcblisting}



%----------------------------------------------
%=== Lists
\begin{tcblisting}{codedemo,title={\dccodetitle{Lists -- enumerate}{enumerate}}}
\begin{enumerate} \item xxx \item xxx \begin{enumerate} \item yyy \item yyy \begin{enumerate} \item zzz \item zzz \item zzz\end{enumerate} \item yyy\end{enumerate} \item xxx\end{enumerate}
\end{tcblisting}


\begin{tcblisting}{codedemo,title={\dccodetitle{Lists -- itemize}{itemize}}}
\begin{itemize} \item xxx \item xxx \begin{itemize} \item yyy \item yyy \begin{itemize} \item zzz \item zzz \item zzz\end{itemize} \item yyy\end{itemize} \item xxx\end{itemize}
\end{tcblisting}


\begin{tcblisting}{codedemo,title={\dccodetitle{Lists -- description}{description}}}
\begin{description} \item[doe] `a deer, a female deer' \begin{description} \item[buck] the male \item[fawn] a young deer \end{description} \item[ray] `a beam of golden sun' \end{description}
\end{tcblisting}


\begin{tcblisting}{codedemo,title={\dccodetitle{Section levels -- section}{section}}}
\section{X} xxx \section*{Y} yyy \section{Z} zzz 
\end{tcblisting}

\section{xxx}

\begin{tcblisting}{codedemo,title={\dccodetitle{Environments -- abstract}{abstract}}}
\begin{abstract}
... x  x  x    x x  x
\end{abstract}
\end{tcblisting}


\getclGen{local} is an example of locally-grouped formatting.



\end{document}